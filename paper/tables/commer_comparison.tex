\begin{table}
\centering
    \small
    \begin{tabular}[htb]{| p{2cm} | p{2.5cm} | p{3cm} | p{3.75cm} | p{2.25cm} |}
        \hline
         & \textbf{Mesh} & \textbf{Timestepping} & \textbf{Compute Resources} & \textbf{Compute Time} \\
        \hline
        Commer FE & 8 421 559 tetrahedral elements & 893 time steps \newline 9 factorizations & single core \newline  Intel Xeon X5550 (2.67 GHz) & 63 hours\\
        \hline
        Commer FD & 2 182 528 cells & $\Delta t = 3 \times 10^{-10}$ s \newline 120 598 277 time-steps & 512 cores \newline Intel Xeon (2.33 GHz) & 23.2 hours\\
        \hline
        UBC OcTree & 5 011 924 cell & 154 time steps \newline  10 factorizations &  single core \newline Intel Xeon X5660 (2.80 GHz) & 57 minutes\\
        \hline
        SimPEG & 314 272 cells & 187 time-steps  \newline 7 factorizations & single core \newline Intel Xeon X5660 (2.80GHz) & 14 minutes \\
        \hline
    \end{tabular}
    \caption{
        Simulation details for the results shown in Figure \ref{fig:commer_results}.
        Note that the discretizations in the Commer FE and FD codes use one element
        or one cell across the width of the casing, as does the UBC code.
        The SimPEG simulation uses 4 cells across the width of the casing.
        For the time-stepping, each chance in step length requires a matrix factorization.
        Values for the Commer FE and FD solutions are from \cite{Commer2015, Um2015}.
    }
    \label{tab:commer_comparison}
 \end{table}


